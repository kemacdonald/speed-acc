\documentclass[english,floatsintext,man]{apa6}

\usepackage{amssymb,amsmath}
\usepackage{ifxetex,ifluatex}
\usepackage{fixltx2e} % provides \textsubscript
\ifnum 0\ifxetex 1\fi\ifluatex 1\fi=0 % if pdftex
  \usepackage[T1]{fontenc}
  \usepackage[utf8]{inputenc}
\else % if luatex or xelatex
  \ifxetex
    \usepackage{mathspec}
    \usepackage{xltxtra,xunicode}
  \else
    \usepackage{fontspec}
  \fi
  \defaultfontfeatures{Mapping=tex-text,Scale=MatchLowercase}
  \newcommand{\euro}{€}
\fi
% use upquote if available, for straight quotes in verbatim environments
\IfFileExists{upquote.sty}{\usepackage{upquote}}{}
% use microtype if available
\IfFileExists{microtype.sty}{\usepackage{microtype}}{}

% Table formatting
\usepackage{longtable, booktabs}
\usepackage{lscape}
% \usepackage[counterclockwise]{rotating}   % Landscape page setup for large tables
\usepackage{multirow}		% Table styling
\usepackage{tabularx}		% Control Column width
\usepackage[flushleft]{threeparttable}	% Allows for three part tables with a specified notes section
\usepackage{threeparttablex}            % Lets threeparttable work with longtable

% Create new environments so endfloat can handle them
% \newenvironment{ltable}
%   {\begin{landscape}\begin{center}\begin{threeparttable}}
%   {\end{threeparttable}\end{center}\end{landscape}}

\newenvironment{lltable}
  {\begin{landscape}\begin{center}\begin{ThreePartTable}}
  {\end{ThreePartTable}\end{center}\end{landscape}}




% The following enables adjusting longtable caption width to table width
% Solution found at http://golatex.de/longtable-mit-caption-so-breit-wie-die-tabelle-t15767.html
\makeatletter
\newcommand\LastLTentrywidth{1em}
\newlength\longtablewidth
\setlength{\longtablewidth}{1in}
\newcommand\getlongtablewidth{%
 \begingroup
  \ifcsname LT@\roman{LT@tables}\endcsname
  \global\longtablewidth=0pt
  \renewcommand\LT@entry[2]{\global\advance\longtablewidth by ##2\relax\gdef\LastLTentrywidth{##2}}%
  \@nameuse{LT@\roman{LT@tables}}%
  \fi
\endgroup}


\ifxetex
  \usepackage[setpagesize=false, % page size defined by xetex
              unicode=false, % unicode breaks when used with xetex
              xetex]{hyperref}
\else
  \usepackage[unicode=true]{hyperref}
\fi
\hypersetup{breaklinks=true,
            pdfauthor={},
            pdftitle={An information-seeking account of children's eye movements during grounded signed and spoken language comprehension},
            colorlinks=true,
            citecolor=blue,
            urlcolor=blue,
            linkcolor=black,
            pdfborder={0 0 0}}
\urlstyle{same}  % don't use monospace font for urls

\setlength{\parindent}{0pt}
%\setlength{\parskip}{0pt plus 0pt minus 0pt}

\setlength{\emergencystretch}{3em}  % prevent overfull lines

\ifxetex
  \usepackage{polyglossia}
  \setmainlanguage{}
\else
  \usepackage[english]{babel}
\fi

% Manuscript styling
\captionsetup{font=singlespacing,justification=justified}
\usepackage{csquotes}
\usepackage{upgreek}

 % Line numbering
  \usepackage{lineno}
  \linenumbers


\usepackage{tikz} % Variable definition to generate author note

% fix for \tightlist problem in pandoc 1.14
\providecommand{\tightlist}{%
  \setlength{\itemsep}{0pt}\setlength{\parskip}{0pt}}

% Essential manuscript parts
  \title{An information-seeking account of children's eye movements during
grounded signed and spoken language comprehension}

  \shorttitle{Information-seeking eye movements}


  \author{Kyle MacDonald\textsuperscript{1}, Virginia Marchman\textsuperscript{1}, Anne Fernald\textsuperscript{1}, \& Michael C. Frank\textsuperscript{1}}

  % \def\affdep{{"", "", "", ""}}%
  % \def\affcity{{"", "", "", ""}}%

  \affiliation{
    \vspace{0.5cm}
          \textsuperscript{1} Stanford University  }

  \authornote{
    Add complete departmental affiliations for each author here. Each new
    line herein must be indented, like this line.
    
    Enter author note here.
    
    Correspondence concerning this article should be addressed to Kyle
    MacDonald, 450 Serra Mall, Stanford, CA 94306. E-mail:
    \href{mailto:kylem4@stanford.edu}{\nolinkurl{kylem4@stanford.edu}}
  }


  \abstract{Language comprehension in grounded, social interactions involves
extracting meaning from the linguistic signal and mapping it to the
visual world. Information that is gathered through visual fixations can
facilitate this comprehension process. But how do listeners decide what
visual information to gather and at what time? Here, we propose that
listeners flexibly adapt their gaze behaviors to seek visual information
from their social partners to support robust language understanding. We
present evidence for our account using three case studies of eye
movements during real-time language processing. First, compared to
children (n=80) and adults (n=25) learning spoken English, young
ASL-learners (n=30) and adults (n= 16) delayed gaze shifts away from a
language source, were more accurate with these shifts, and produced a
smaller proportion of random shifts. Next, English-speaking adults
(n=30) produced fewer random gaze shifts when processing serially
printed text compared to processing spoken language. Finally,
English-speaking preschoolers (n=39) and adults (n=31) delayed the
timing of shifts away from a speaker while processing speech in noisy
environments, gathering more visual information while generating more
accurate responses. Together, these results provide evidence that
listeners adapt to the demands of different processing contexts by
seeking out visual information from social partners.}
  \keywords{eye movements; language comprehension; information-seeking; speech in
background noise; American Sign Language \\

    \indent Word count: X
  }





\begin{document}

\maketitle

\setcounter{secnumdepth}{0}



\hypertarget{introduction}{%
\section{Introduction}\label{introduction}}

Extracting meaning from language represents a formidable challenge for
young language learners. Consider that even in the simple case of
understanding grounded, familiar language (e.g., \enquote{look at the
ball}), the listener must continuously integrate linguistic and
non-linguistic information from continuous streams of input. Moreover,
language unfolds within dynamic interactions where there is often
insufficient information to figure out what is being said, and yet the
listener must decide how best to respond. Even young children, however,
can map language to the world quite efficiently, shifting visual
attention to a named object in a scene within hundreds of milliseconds
upon hearing the name of an object (Allopenna, Magnuson, \& Tanenhaus,
1998; Spivey, Tanenhaus, Eberhard, \& Sedivy, 2002; Tanenhaus,
Spivey-Knowlton, Eberhard, \& Sedivy, 1995). How do young listeners
successfully interpret linguistic input despite these challenges?

One solution is for the language comprehension system to integrate
multiple sources of information to constrain the set of possible
interpretations (MacDonald \& Seidenberg, 2006; McClelland \& Elman,
1986). Under this interactive account, listeners comprehend words by
partially activating several candidates that are consistent with the
incoming perceptual information. Then, as more information arrives,
words that do not match the perceptual signal are no longer considered,
and words that are more consistent become strongly activated until a
single interpretation is reached (see McClelland, Mirman, and Holt
(2006) for a review). Critically, information from multiple sources --
e.g., the linguistic signal and visual world -- mutually influence one
another to shape interpretation. For example, if a speaker's mouth
movements suggest one sound while their acoustic output indicates
another, the conflict results in the listener perceiving a third,
intermediate sound (\enquote{McGurk effect}) (MacDonald \& McGurk,
1978). Other research shows that listeners will use information in the
visual scene to help parse syntactically ambiguous utterances (Tanenhaus
et al., 1995).

Thus, information gathered from the visual world can serve as a useful
constraint on language comprehension. But the incoming linguistic
information is ephemeral, meaning listeners must quickly decide how to
direct their gaze to informative locations. Consider a speaker who asks
you to \enquote{Pass the salt} in a noisy restaurant. Here,
comprehension could be supported by looks that better encode the objects
in the scene (e.g., the type of food she is eating), or by looks to the
speaker (e.g., reading her lips or the direction of her gaze). A second
interesting case is the processing a visual-manual language such as
American Sign Language (ASL). Here, deciding to look away from another
signer is inherently risky because the listener stops the flow of
information from the linguistic signal to gather information about the
visual world.

Eye movements during language comprehension can be characterized an
active decision-making process taking into account limits on visual
attention. We propose that listeners are sensitive to this tradeoff,
flexibly adapting the dynamics of their gaze in contexts that place a
higher value on gathering visual information. That is, we suggest that
listeners' eye movements are shaped by an interaction between their
sensorimotor constraints and information features of the environment.

Our account is inspired by ideas from several research traditions.
First, work on language-mediated visual attention showing rapid
interactions between visual attention and language (Allopenna et al.,
1998; Tanenhaus et al., 1995). Second, research on vision in everyday
tasks shows that people allocate fixations to \emph{goal-relevant}
locations -- e.g., an upcoming obstacle while walking (Hayhoe \&
Ballard, 2005). Finally, work on multisensory integration showing that
listeners leverage multimodal cues (e.g., gestures, facial expressions,
mouth movements) to support communication. In the following sections, we
review each of these literatures to motivate our information-seeking
account of eye movements in social, grounded language comprehension.

\hypertarget{vision-language-interactions-during-language-comprehension}{%
\subsection{Vision-Language interactions during language
comprehension}\label{vision-language-interactions-during-language-comprehension}}

Eye movements during language comprehension have provided insight into
the interaction between concepts, language, and visual attention. The
majority of this work has used the Visual World Paradigm (VWP) where
listeners' eye movements are recorded at the millisecond timescale while
processing language and looking at a set of objects (see Salverda,
Brown, and Tanenhaus (2011) for a review). Crucially, these analyses
rely on the fact that people will initiate gaze shifts to naemd
referents with only partial information, in contrast to waiting until
the end of a cognitive process (Gold \& Shadlen, 2000). Thus, the
timecourse of eye movements provides a window onto how and when people
integrate information to reach an interpretation of the incoming
linguistic signal.

A classic finding using the VWP shows that listeners will rapidly shift
visual attention upon hearing the name of an object (\enquote{Pick up a
beaker.}) in the visual scene with a high proportion of shifts occurring
soon after the target word begins (Allopenna et al., 1998). Moreover,
adults tended to look at phonological onset-competitor (a beetle) early
in the target noun, suggesting that they had activated multiple
interpretations and resolved ambiguity as the stimulus unfolded. These
behavioral results fall out of predictions made by interactive models of
speech perception where information from multiple sources is integrated
to constrain language understanding (McClelland et al., 2006).

The visual world can also constrain the set of plausible interpretations
of language (Dahan \& Tanenhaus, 2005; Yee \& Sedivy, 2006). For
example, Altmann and Kamide (2007) showed that people will allocate more
looks to an empty wine glass as compared to a full beer glass upon
hearing the past tense verb \enquote{has drunk.} They propose that
anticipatory eye movements reflect the influence of the visual
information in a scene activating a multi-featured, conceptual
representation prior to the arrival of the linguistic signal (see also
Huettig and Altmann (2005)).

In addition to work on adult psycholinguistics, the VWP has been useful
for studying developmental change in language comprehension skill in
children. Researchers have adapted the task to measure the timing and
accuracy of children's gaze shifts as they look at two familiar objects
and listen to simple sentences naming one of the objects (Fernald,
Zangl, Portillo, \& Marchman, 2008; Venker, Eernisse, Saffran, \&
Weismer, 2013). Such research finds that children, like adults, shift
gaze to named objects occur soon after the auditory information is
sufficient to enable referent identification. Moreover, individual
differences in the speed and accuracy of eye movements predict
vocabulary growth and later language and cognitive outcomes (Fernald,
Perfors, \& Marchman, 2006; Marchman \& Fernald, 2008; Rigler et al.,
2015). Finally, the VWP illustrated interesting developmental parallels
and differences between children's language processing in different
populations, including sign language (MacDonald, LaMarr, Corina,
Marchman, \& Fernald, 2018), bilingualism (Byers-Heinlein,
Morin-Lessard, \& Lew-Williams, 2017), and children with cochlear
implants (Schwartz, Steinman, Ying, Mystal, \& Houston, 2013).

\hypertarget{goal-based-accounts-of-eye-movements-in-everyday-tasks}{%
\subsection{Goal-based accounts of eye movements in everyday
tasks}\label{goal-based-accounts-of-eye-movements-in-everyday-tasks}}

The majority of the work on language-mediated visual attention has used
eye movements as an index of the underlying interaction between
linguistic and visual information. This approach reflects a somewhat
passive construal of how people allocate visual attention during
language comprehension. In contrast, goal-based accounts of vision start
from the idea that eye movements reflect an active information-gathering
process where visual fixations are driven by task goals (Hayhoe \&
Ballard, 2005).

Under this account, people allocate visual attention to reduce
uncertainty about the world and maximize the expected future reward. For
example, Hayhoe and Ballard (2005) review evidence that people fixate on
locations that are most helpful for their current goal (an upcoming
obstacle) as opposed to other aspects of a visual scene that might be
more salient (a flashing light). Moreover, other work shows that people
gather task-specific information via different visual routines as they
become useful for their goals. For example, Triesch et al 2003 found
that people were much less likely to gather and store visual information
about the size of an object when it was not relevant to the task of
sorting and stacking the objects.

Hayhoe and Ballard (2005)'s review also highlights the role of learning
gaze patterns. They point out that visual routines are developed over
time, and it is only when a task becomes highly-practiced that people
allocate fewer looks to less relevant parts of the scene. For example,
Shinoda, Hayhoe, and Shrivastava (2001) show that drivers, with
practice, learn to spread visual attention more broadly at intersections
to better detect stop signs. Other empirical work shows that the visual
system rapidly learns to use temporal regularities in the environment to
control the timing of eye movements to detect goal-relevant events
(Hoppe \& Rothkopf, 2016). Moreover, the timing of eye movements in
these tasks often occur before an expected event (i.e., anticipatory),
suggesting that gaze patterns reflect an interaction between people's
expectations, information available in the visual scene, and their task
goals.

Recent theoretical work has argued for a stronger link between
goal-based perspectives and work on eye movements during language
comprehension. For example, Salverda et al. (2011) highlight the
immediate relevance of visual information with respect to the goal of
language understanding, suggesting that listeners' goals should be a key
predictor of fixation patterns. Moreover, they point out that factors
such as the difficulty of executing a real world task should change
decisions about where to look during language comprehension. One example
of starting from a goal-based approach comes from Nelson and Cottrell
(2007)' study of gaze patterns during category learning. Nelson and
Cottrell (2007) modeled eye movements as a type of question-asking about
features of a concept and showed that the dynamics of eye movements
changed as participants became more familiar with the novel concepts.
Early in learning people generated a broader distribution of fixations
to explore all features. Later in learning, eye movements shifted to
focus on a single stimulus dimension to maximize accuracy on the task.
This shift in gaze patterns, from exploratory to efficient, suggests
that fixation behavior changed as a function of changes in learning
goals during the experiment.

In the current studies, the goal-based model of eye movements predicts
that gaze dynamics during language comprehension should adapt to the
processing context. That is, listeners should change the timing and
location of eye movements when fixation locations become more useful for
language understanding. This proposal dovetails with a growing body of
research that explores the role of multisensory information available in
face-to-face communication such as gesture, prosody, facial expression
and body movement.

\hypertarget{language-perception-as-multisensory-integration}{%
\subsection{Language perception as multisensory
integration}\label{language-perception-as-multisensory-integration}}

Language comprehension is not just one stream of linguistic information.
Instead, face-to-face communication provides access to a set of
multimodal cues that can facilitate comprehension and there is a growing
emphasis on studying language as a multimodal and multisensory process
(for a review, see Vigliocco, Perniss, and Vinson (2014)). For example,
empirical work shows that when gesture and speech provide redundant cues
to meaning, people are faster to process the information and make fewer
errors (Kelly, Özyürek, \& Maris, 2010). Moreover, developmental work
shows that parents use visual cues such as gesture and eye gaze to help
structure language interactions with their children (Estigarribia \&
Clark, 2007). Finally, from a young age, children also produce gestures
such as reaches and points to share attention with others to achieve
communicative goals (Liszkowski, Brown, Callaghan, Takada, \& De Vos,
2012).

Additional support for multisensory processing comes from work on
audiovisual speech perception, showing how spoken language perception is
shaped by visual information coming from a speaker's mouth. In a review,
Peelle and Sommers (2015) point out that mouth movements provide a clear
indication of when someone has started to speak, which cues the listener
to allocate additional attention to the speech signal. Moreover, a
speaker's mouth movements convey information about the phonemes in the
acoustic signal. For example, visual speech information distinguishes
between consonants such as /b/ vs. /d/ and place of articulation can
help a listener differentiate between words such as \enquote{cat} or
\enquote{cap.} Finally, classic empirical work shows comprehension
benefits for audiovisual speech compared to auditory- or visual-only
speech, especially in noisy listening contexts (Erber, 1969).

In sum, the work on multisensory processing shows that both auditory and
visual information interact to shape language perception. These results
dovetail with the interactive models of language processing reviewed
earlier and suggest that visual information can support comprehension
(MacDonald \& Seidenberg, 2006; McClelland et al., 2006). Finally, these
results highlight the value of studying language comprehension during
face-to-face communication, where listeners have the choice to gather
visual information about the linguistic signal from their social
partners.

\hypertarget{the-present-studies}{%
\subsection{The present studies}\label{the-present-studies}}

The studies reported here synthesize ideas from research on language
processing as a multimodal, goal-based, and social phenomenon. We
propose an information-seeking account of eye movements during grounded
language comprehension in social interaction. We characterize the timing
of gaze patterns as reflecting a tradeoff between gathering visual
information about the incoming linguistic signal from a speaker and
seeking information about the surrounding visual scene. We draw on
models of eye movements as active decisions that gather information to
achieve reliable interpretations of incoming language. We test
predictions of our account using three case studies: sign language, text
processing, and processing spoken language in noisy environments. These
case studies represent a broad sampling of contexts that share a key
feature: The interaction between the listener and context changes the
value of fixating on the language source to gather visual information
for comprehension.

A secondary goal of this work was to test whether children and adults
would show similar patterns of adaptation of gaze patterns. Recent
developmental work shows that, like adults, preschoolers will flexibly
adjust how they interpret ambiguous sentences (e.g., \enquote{I had
carrots and \emph{bees} for dinner.}) by integrating information about
the reliability of the incoming perceptual information with their
expectations about the speaker (Yurovsky, Case, \& Frank, 2017). While
children's behavior paralleled adults, they relied more on top-down
expectations about the speaker perhaps because their perceptual
representations were noisier. These developmental differences provide
insight into how children succeed in understanding language despite
having partial knowledge of word-object mappings.

The structure of the paper is as follows. First, we compare children and
adult's eye movements while processing signed vs.~spoken language. Then,
we present a comparison of adults' eye movements while processing
serially printed text vs.~spoken language. Finally, we compare children
and adults' gaze patterns while they process speech in noisy vs.~clear
auditory environments. The key behavioral prediction (see Table 1 for
more detailed predictions) is that both children and adults will adapt
the timing of their eye movements to facilitate better word recognition.
We hypothesized that when a language source provides higher value visual
information, listeners would prioritize fixations to their social
partner, and, in turn, would be slower to shift gaze away, generating
(a) more accurate responses and (b) fewer random, exploratory eye
movements to the objects in the visual scene.

Before describing the studies, it is worth motivating our analytic
approach. To quantify the evidence for our predictions, we analyze the
accuracy and reaction times (RTs) of listeners' initial gaze shifts
after hearing the name of an object. The timescale of this analysis is
milliseconds and focuses on a single decision within a series of
decisions about where to look during language processing. We chose this
approach because first shifts are rapid decisions driven by accumulating
information about the identity of the named object. Moreover, these
measurements provide a window onto changes in the underlying dynamics of
how listeners integrate linguistic and visual information to make
fixation decisions. Finally, by focusing our analysis on a specific
decision, we could leverage models of decision making developed over the
past decades to quantify changes in the underlying dynamics of eye
movements in different processing contexts (see the analysis plan for
more details).

\hypertarget{experiment-1}{%
\section{Experiment 1}\label{experiment-1}}

Experiment 1 provides an initial test of our adaptive tradeoffs account.
We compared eye movements of children learning American Sign Language to
children learning a spoken language using parallel real-time language
comprehension tasks where children processed familiar sentences (e.g.,
\enquote{Where's the ball?}) while looking at a simplified visual world
with 3 fixation targets (a center stimulus that varied by condition, a
target picture, and a distracter picture; see Fig 1). The spoken
language data are a reanalysis of three unpublished data sets, and the
ASL data are reported in MacDonald et al. (2018). We predicted that,
compared to spoken language processing, processing ASL would increase
the value of fixating on the language source and decrease the value of
generating exploratory, nonlanguage-driven shifts even after the
disambiguating point in the linguistic signal.

\hypertarget{analysis-plan}{%
\subsection{Analysis plan}\label{analysis-plan}}

First, we present behavioral analyses of First Shift Accuracy and
Reaction Time (RT). RT corresponds to the latency to shift away from the
central stimulus to either picture measured from the onset of the target
noun. Accuracy corresponds to whether participants' first gaze shift
landed on the target or the distracter picture. It is important to note
that this analysis of accuracy does not measure the overall amount of
time spent looking at the target vs.~the distractor image -- a measure
typically used in analyses of the Visual World Paradigm. We chose to
focus on first shifts to provide a window onto how processing contexts
change the underlying dynamics of information gathering decisions. All
analysis code can be found in the online repository for this project:
\url{https://github.com/kemacdonald/speed-acc}.

We used the \texttt{rstanarm} (Gabry \& Goodrich, 2016) package to fit
Bayesian mixed-effects regression models. The mixed-effects approach
allowed us to model the nested structure of our data -- multiple trials
for each participant and item, and a within-participants manipulation --
by including random intercepts for each participant and item, and a
random slope for each item and noise condition. We used Bayesian
estimation to quantify uncertainty in our point estimates, which we
communicate using a 95\% Highest Density Interval (HDI). The HDI
provides a range of credible values given the data and model. Finally,
to estimate age-related differences, we fit two types of models: (1) age
group (adults vs.~children) as a categorical predictor and (2) age as a
continuous predictor (measured in days) within the child sample.

Next, we present the two model-based analyses -- the EWMA and DDM. The
goal of these models is to move beyond a description of the data and map
behavioral differences in eye movements to underlying psychological
variables. The EWMA method models changes in random shifting behavior as
a function of RT. For each RT, the model generates two values: a
\enquote{control statistic} (CS, which captures the running average
accuracy of first shifts) and an \enquote{upper control limit} (UCL,
which captures the pre-defined limit of when accuracy would be
categorized as above chance level). Here, the CS is an expectation of
random shifting to either the target or the distracter image
(nonlanguage-driven shifts), or a Bernoulli process with probability of
success 0.5. As RTs get slower, we assume that participants have
gathered more information and should become more accurate
(language-driven), or a Bernoulli process with probability success
\textgreater{} 0.5. Using this model, we can quantify the proportion of
gaze shifts that were language-driven as opposed to random responding.

Following Vandekerckhove and Tuerlinckx (2007), we selected shifts
categorized as language-driven by the EWMA and fit a hierarchical
Bayesian drift-diffusion model (HDDM). The DDM quantifies differences in
the underlying decision process that lead to different patterns of
behavior. The model assumes that people accumulate noisy evidence in
favor of one alternative with a response generated when the evidence
crosses a pre-defined decision threshold. We chose to implement a
hierarchical Bayesian version of the DDM using the HDDM Python package
(Wiecki, Sofer, \& Frank, 2013) since we had relatively few trials from
child participants and recent simulation studies have shown that the
HDDM approach was better than other DDM fitting methods for small data
sets (Ratcliff \& Childers, 2015). Here, we focus on two parameters of
interest: \emph{boundary separation}, which indexes the amount of
evidence gathered before generating a response (higher values suggest
more cautious responding) and \emph{drift rate}, which indexes the
amount of evidence accumulated per unit time (higher values suggest more
efficient processing).

\hypertarget{methods}{%
\subsection{Methods}\label{methods}}

\hypertarget{participants}{%
\subsubsection{Participants}\label{participants}}

\begin{table}[tbp]
\begin{center}
\begin{threeparttable}
\caption{\label{tab:trio make participants table}Age distributions of children in Experiment 1. All ages are reported in months.}
\begin{tabular}{lllll}
\toprule
Center Stimulus & \multicolumn{1}{c}{Mean} & \multicolumn{1}{c}{Min} & \multicolumn{1}{c}{Max} & \multicolumn{1}{c}{n}\\
\midrule
ASL & 27.90 & 16.00 & 53.00 & 30.00\\
Face & 26.00 & 25.00 & 26.00 & 24.00\\
Object & 31.90 & 26.00 & 39.00 & 40.00\\
Bullseye & 26.10 & 26.00 & 27.00 & 16.00\\
\bottomrule
\end{tabular}
\end{threeparttable}
\end{center}
\end{table}

Table 1 contains details about the age distributions of children in all
of four samples.

\emph{Spoken English samples.} Participants were 80 native, monolingual
English-learning children divided across three samples. Participants had
no reported history of developmental or language delay.

\emph{ASL sample.} Participants were 30 native, monolingual ASL-learning
children (18 deaf, 12 hearing). All children, regardless of hearing
status, were exposed to ASL from birth through extensive interaction
with at least one caregiver fluent in ASL and were reported to
experience at least 80\% ASL in their daily lives. The ASL sample
included a wider age range compared to the spoken English samples
because this is a rare population.

\hypertarget{stimuli}{%
\subsubsection{Stimuli}\label{stimuli}}

\begin{figure}[!t]

{\centering \includegraphics[width=0.85\linewidth]{/Users/kmacdonald/Documents/Projects/SPEED-ACC/paper/journal_submission/figures/figs_output//trio_stimuli} 

}

\caption{Stimuli for Experiment 1. Panel A shows the timecourse of the linguistic stimuli for a single trial. Panel B shows the layout of the fixation locations for all tasks: the center stimulus, the target, and the distracter. Panel C shows the four center stimulus items: a static geometric shape (Bullseye), a static image of a familiar object (Object), a person speaking (Face), and a person signing (ASL).}\label{fig:trio-stim}
\end{figure}

There are differences between ASL and English question structures.
However, all linguistic stimuli shared the same trial structure:
language to attract participants' attention followed by a sentence
containing a target noun.

\emph{ASL linguistic stimuli.} We recorded two sets of ASL stimuli,
using two valid ASL sentence structures for questions: 1)
Sentence-initial wh-phrase: \enquote{HEY! WHERE {[}target noun{]}?} and
2) Sentence-final wh-phrase: \enquote{HEY! {[}target noun{]} WHERE?} Two
female native ASL users recorded several tokens of each sentence in a
child-directed register. Before each sentence, the signer produced a
common attention-getting gesture. Mean sign length was 1.25 sec, ranging
from 0.69 sec to 1.98 sec.

\emph{English linguistic stimuli.} All three tasks (Object, Bullseye,
and Face) featured the same female speaker who used natural
child-directed speech and said: \enquote{Look! Where's the (target
word)?} The target words were: ball, banana, book, cookie, juice, and
shoe. For the Face task, a female native English speaker was
video-recorded as she looked straight ahead and said, \enquote{Look!
Where's the (target word)?} Mean word length was 0.79 sec, ranging from
0.60 sec to 0.94 sec.

\emph{ASL and English visual stimuli.} The image set consisted of
colorful digitized pictures of objects presented in fixed pairs with no
phonological overlap (ASL task: cat---bird, car---book, bear---doll,
ball---shoe; English tasks: book-shoe, juice-banana, cookie-ball). Side
of target picture was counterbalanced across trials.

\emph{Trial structure.} On each trial, the child saw two images of
familiar objects on the screen for two seconds before the center
stimulus appeared. This time allowed the child to visually explore both
images. Next, the target sentence -- which consisted of a carrier
phrase, target noun, and question sign -- was presented, followed by two
seconds without language to allow the child to respond to the signer's
sentence. The trial structure of the Face, Object, and Bullseye tasks
were highly similar: children were given two seconds to visually explore
the objects prior to the appearance of the center stimulus, then
processed a target sentence, and finally were given two seconds of
silence to generate a response to the target noun.

\hypertarget{design-and-procedure}{%
\subsubsection{Design and procedure}\label{design-and-procedure}}

Children sat on their caregiver's lap and viewed the task on a screen
while their gaze was recorded using a digital camcorder. On each trial,
children saw two images of familiar objects on the screen for two
seconds before the center stimulus appeared (see Figure 1). Then they
processed the target sentence -- which consisted of a carrier phrase, a
target noun, and a question -- followed by two seconds without language
to allow for a response. Participants saw 32 test trials with several
filler trials interspersed to maintain interest.

\emph{Coding.} Participants' gaze patterns were coded (33-ms resolution)
as being fixated on either the center stimulus, one of the images,
shifting between pictures, or away. To assess inter-coder reliability,
25\% of the videos were re-coded. Agreement was scored at the level of
individual frames of video and averaged 98\% on these reliability
assessments.

\hypertarget{results-and-discussion}{%
\subsection{Results and Discussion}\label{results-and-discussion}}

\begin{figure}[!t]

{\centering \includegraphics[width=0.85\linewidth]{/Users/kmacdonald/Documents/Projects/SPEED-ACC/paper/journal_submission/figures/figs_output//fig1_trio_behav} 

}

\caption{Timecourse looking, first shift Reaction Time (RT), and Accuracy results for children in Experiment 1. Panel A shows the overall looking to the center, target, and distracter stimulus for each context. Panel B shows the distribution of RTs for each participant. Each point represents a participant's average RT. Color represents the processing context. Panel C shows the same information but for first shift accuracy.}\label{fig:speed-acc-trio-plot}
\end{figure}

\hypertarget{behavioral-analyses}{%
\subsubsection{Behavioral analyses}\label{behavioral-analyses}}

\emph{Time course looking.} The first question of interest was how do
young ASL and English learners distribute attention while processing
language in real-time? Panel A of Fig.~\ref{fig:speed-acc-trio-plot}
presents an overview of children's looking to the center stimulus,
target, and distractor images for each processing context. This plot
shows changes in the mean proportion of trials on which participants
fixated the signer, the target image, or the distracter image at every
33-ms interval of the stimulus sentence. At target-noun onset, children
were looking at the center on all trials. As the target noun unfolded,
the mean proportion looking to the center decreased rapidly as
participants shifted their gaze to the target or the distracter image.
Proportion looking to the target increased sooner and reached a higher
asymptote, compared to proportion looking to the distracter, for all
four processing contexts.

After looking to the target image, participants tended to shift their
gaze back to the center, shown by the increase in proportion looking to
the center around two seconds after target-noun onset. There were
several qualitative differences in children's looking behavior across
the different center stimulus types. First, ASL-learners and
English-learners who processed sentences from a video of speaker spent
more time looking to the center as indicated by the shallower slope on
their center-looking curves. Second, when the center stimulus was a
static geometric object (Bullseye) or a static familiar object (Object),
spoken language learners were more likely to look at the distractor
image, especially early in the time course of the target noun as
indicated by the parallel increase in target and distractor-looking
curves in Fig. ~\ref{fig:speed-acc-trio-plot}. In contrast, spoken
language learners in the Face context spent less time looking at the
disracter, and ASL-learners rarely looked to the distractor image at any
point in the trial. This pattern of behavior provides qualitative
evidence that the children were adpating the dynamics of their looking
depending on the nature of the visual world and language modality.

\emph{RT.} Panel B of Fig.~\ref{fig:speed-acc-trio-plot} shows the full
RT data distribution and the full posterior distribution of the
estimated RT difference between the noise and clear conditions. To
quantify differences across the groups, we fit a Bayesian linear
mixed-effects regression predicting first shift RT as a function of
center stimulus type, controlling for age, and including user-defined
contrasts to test specific comparisons of interest:
\texttt{Log(RT) $\sim$ center stimulus type + age +  (1 | subject) + (1 | item)}.
We found that (a) ASL learners generated slower RTs compared to all of
the spoken English samples (\(\beta\) = 0.60 sec, 95\% HDI {[}0.44 sec,
0.76 sec{]}), (b) ASL learners' shifts were slower compared directly to
children processing spoken language in the Face condition (\(\beta\) =
0.32 sec, 95\% HDI {[}0.13 sec, 0.52 sec{]}), and (c) children in the
Face condition shifted gaze slower compared to participants in the
Object and Bullseye tasks (\(\beta\) = 0.41 sec, 95\% HDI {[}0.29 sec,
0.55 sec{]}).

\emph{Accuracy.} Next we compared the accuracy of first shifts across
the different tasks by fitting a mixed-effects logistic regression with
the same specifications and contrasts as the RT model. We found that (a)
ASL learners were more accurate compared to all of the spoken English
samples (\(\beta\) = 0.23 sec, 95\% HDI {[}0.17{]}, 0.29), (b) ASL
learners were more accurate when directly compared to participants in
the Face task (\(\beta\) = 0.13 sec, 95\% HDI {[}0.04 sec, 0.23 sec{]}),
(c) children learning spoken language were more accurate when processing
language from dynamic video of a person speaking compared to the Object
and Bullseye tasks (\(\beta\) = 0.16 sec, 95\% HDI {[}0.07 sec, 0.24
sec{]}), and (d) English-learners' first shifts were no different from
random responding in the Object (\(\beta\) = -0.04 sec, 95\% HDI
{[}-0.13 sec, 0.03 sec{]}) and Bullseye (\(\beta\) = -0.02 sec, 95\% HDI
{[}-0.12 sec, 0.08 sec{]}) contexts.

\hypertarget{model-based-analyses}{%
\subsubsection{Model-based analyses}\label{model-based-analyses}}

\begin{figure}[!t]

{\centering \includegraphics[width=0.85\linewidth]{/Users/kmacdonald/Documents/Projects/SPEED-ACC/paper/journal_submission/figures/figs_output//fig2_trio_models} 

}

\caption{Results for the model-based analyses in Experiment 1. Panel A shows a control chart representing the timecourse of the EWMA statistic. The black curve represents the evolution of the control statistic (CS) as a function of reaction time. The grey curve represents the upper control limit (UCL). The vertical dashed line is the median cutoff value (point when the control process shifts out of a guessing state). The grey shaded area represents the 95\% confidence interval around the estimate of the median cutoff point, and the shaded ribbons represent the proportion of responses that were categorized as guesses (red) and language-driven (green). Panel B shows a summary of the proportion of shifts that were categorized as language-driven for the Face and ASL processing contexts. Panel C shows the posterior distributions for the boundary and drift rate parameters for the Face and ASL processing contexts.}\label{fig:trio-model-plot}
\end{figure}

\emph{EWMA.} Panel A of Fig.~\ref{fig:trio-model-plot} shows changes in
the control statistic (CS) and the upper control limit (UCL) as a
function of RT. Each CS starts at chance performance and below the UCL.
In the ASL and Face tasks, the CS value begins to increase with RTs
around 0.7 seconds after noun onset and eventually crosses the UCL,
indicating that responses \textgreater{} 0.7 sec were on average above
chance levels. In contrast, the CS in the Object and Bullseye tasks
never crossed the UCL, indicating that children's shifts were equally
likely to land on the target or the distracter, regardless of when they
were initiated. This result suggests that first shifts in the
Bullseye/Object tasks were not language-driven and may instead have
reflected a different process such as gathering more information about
the referents in the visual world.

Next, we compared the EWMA model fits for participants in the ASL and
Face processing contexts. We found that ASL learners generated fewer
shifts when the CS was below the UCL compared to children learning
spoken language (\(\beta\) = 0.14, 95\% HDI {[}0.08, 0.23{]}). This
result indicates that ASL-learners were more likely to have gathered
sufficient information about the linguistic signal prior to shifting
gaze away from the language source (i.e., gaze shifts were
language-driven). We found some evidence that ASL learners started
producing language-driven shifts earlier in the RT distribution as
indicated by the point at which the CS crossed the UCL (\(\beta\) = 0.22
ms, 95\% HDI {[}0.05 ms, 0.39 ms{]}).

\emph{HDDM.} Using the output of the EWMA, we compared the timing and
accuracy of language-driven shifts for participants in the ASL and Face
tasks.\footnote{We report the mean and the 95\% highest density interval
  (HDI) of the posterior distributions for each parameter. The HDI
  represents the range of credible values given the model specification
  and the data. We chose not to interpret the DDM fits for the
  Bullseye/Face tasks since there was no suggestion of any non-guessing
  signal.} We found that ASL learners had a higher estimate for the
boundary separation parameter compared to the Face participants (ASL
boundary = 1.76, HDI = {[}1.65, 1.88{]}; Face boundary = 1.34, HDI =
{[}1.21, 1.47{]}), with no overlap in the credible values (see Fig 4).
This suggests that ASL learners accumulated more evidence about the
linguistic signal before generating an eye movement. We found high
overlap for estimates of the drift rate parameter, indicating that both
groups processed the linguistic information with similar efficiency (ASL
drift = 0.63, HDI = {[}0.44, 0.82{]}; Face drift = 0.55, HDI = {[}0.30,
0.80{]}).

\emph{Results summary.} Taken together, the behavioral analyses and the
EWMA/HDDM results provide converging support that ASL learners were
sensitive to the value of delaying eye movements away from the language
source. Compared to spoken language learners, children learning ASL
prioritized accuracy over speed, produced fewer nonlanguage-driven
shifts away from the center stimulus, and were more accurate with these
gaze shifts. Importantly, we did not see evidence in the HDDM model fits
that these accuracy differences could be explained by differential rates
of information accumulation. Instead, the model-based analyses suggest
that ASL learners increased their decision threshold for generating a
response.

We hypothesized that this prioritization of gathering additional
information is an adaptive response to the channel competition present
when processing a visual-manual language. That is, when ASL learners
shift gaze away a signer, they are deciding to leave an area of the
visual world that provides a great deal of useful and interesting
information. Moreover, unlike children leanring spoken languages, ASL
learners cannot gather more of the linguistic signal while looking at
the objects. Thus, an adaptive language comprehension system would
increase levels of certainty before generating a response to maintain
robust understanding.

It is important to point out that these findings are based on
exploratory analyses, and our information seeking account was developed
to explain this pattern of results. There are, however, several,
potentially important differences between the stimuli, apparatus, and
populations that limit the sterngth of our interpretation of these data
and the generality of our account. Thus, in Experiments 2 and 3, we set
out to perform well-controlleds, confirmatory tests of our adaptive
information seeking account of eye movements during grounded language
comprehension.

\hypertarget{experiment-2}{%
\section{Experiment 2}\label{experiment-2}}

In Experiment 2, we aimed to replicate a key finding from Experiment 1:
that increasing the competition between fixating the language source and
the nonlinguistic visual world reduces nonlanguage-driven eye movements.
Moreover, we conducted a confirmatory test of our hypothesis that also
controlled for the population differences present in Experiment 1. We
tested a sample of English-speaking adults using a within-participants
manipulation of the center stimulus type. We used the Face and Bullseye
stimulus sets from Experiment 1 and added two new conditions: Text,
where the verbal language information was accompanied by a word-by-word
display of printed text (see Figure 3), and Text-no-audio, where the
spoken language stimulus was removed. We chose text processing since,
like sign language comprehension, information relevant to the linguistic
signal is concentrated in one location in the visual scene.

Our key behavioral prediction is that processing serially-presented text
will shift the value of allocating fixations to the center stimulus as
the linguistic information unfolds in time. This shift in information
value should result in listeners allocating more fixations to the center
stimulus and fewer to the objects in the visual scene. This behavioral
pattern should be indexed by proportion guessing and cutoff point
parameters of the EWMA model. We did not have strong predictions for
first shift accuracy and reaaction time or the DDM parameter fits since
the goal of the text manipulation was to modulate participants'
strategic allocation of visual attention and not the accuracy/efficiency
of information processing.

\hypertarget{methods-1}{%
\subsection{Methods}\label{methods-1}}

\hypertarget{participants-1}{%
\subsubsection{Participants}\label{participants-1}}

25 Stanford undergraduates participated (5 male) for course credit. All
participants were monolingual, native English speakers and had normal
vision.

\hypertarget{stimuli-1}{%
\subsubsection{Stimuli}\label{stimuli-1}}

Audio and visual stimuli were identical to the Face and Bullseye tasks
in Experiment 1. We included a new center fixation stimulus type:
printed text. The text was displayed in a white font on a black
background and was programmed such that only a single word appeared on
the screen, with each word appearing for the same duration as the
corresponding word in the spoken language stimuli.

\hypertarget{design-and-procedure-1}{%
\subsubsection{Design and procedure}\label{design-and-procedure-1}}

The design was nearly identical to Experiment 1, with the exception of a
change to a within-subjects manipulation where each participant
completed all four tasks (Bullseye, Face, Text, and Text-no-audio). In
the Text condition, spoken language accompanied the printed text. In the
Text-no-audio condition, the spoken language stimulus was removed.
Participants saw a total of 128 trials while their eye movements were
tracked using automated eye-tracking software.

\hypertarget{results-and-discussion-1}{%
\subsection{Results and Discussion}\label{results-and-discussion-1}}

\begin{figure}[!t]

{\centering \includegraphics[width=0.85\linewidth]{figs/text-plot-1} 

}

\caption{Results for the model-based analyses in Experiment 2. All plotting conventions are the same as in Figure 2.}\label{fig:text-plot}
\end{figure}

\hypertarget{behavioral-analyses-1}{%
\subsubsection{Behavioral analyses}\label{behavioral-analyses-1}}

\emph{Time course looking.} The first question of interest was how do
young ASL and English learners distribute attention while processing
language in real-time? Panel A of Fig.~\ref{fig:text-plot} presents an
overview of adults's looking to the center stimulus, target, and
distractor images for each center stimulus type. Similar to children's
looking behavior in Experiment 1, at target-noun onset the majority of
adults were looking at the center. As the target noun unfolded, the mean
proportion looking to the center decreased rapidly as participants
shifted their gaze to the objects. Proportion looking to the target
increased sooner and reached a higher asymptote compared to proportion
looking to the distracter for all four processing contexts.

After looking to the target image, adults did not tend to shif their
gaze back to the center as shown by the relatively flat proportion
looking to the center curves one second after target-noun onset. The
primary qualitative difference in adult's looking behavior across the
different center stimulus types was a higher tendency to be looking at
the center stimulus in the Face and Text conditions relative to the
Bullseye condition. This was especially true for the Text-no-audio
condition where adults were looking to the center at target-noun onset
on 100\% of the trials. This pattern of behavior provides preliminary,
qualitative evidence that the condition manipulation affected the
dynamics of adults looking beahvior early in the time course of the
target noun. We did not see evidence of differences in the asymptotes or
slopes for the target- or distracter-looking curves.

\emph{RT.} Visual inspection of Figure 5, panel C suggests that mean
response times of first shifts were similar across the four center
stimulus conditions (\(M_{bull}\) = 0.55 sec, \(M_{face}\) = 0.59 sec,
\(M_{text}\) = 0.58 sec, \(M_{textNoaudio}\) = 0.56 sec). We fit a
linear mixed-effects regression with the same specification as in
Experiment 1, but we added by-subject intercepts and slopes for each
center stimulus type to account for our within-subjects manipulation. We
did not see evidence that RTs were different across conditions, with the
null value of zero condition differences falling within the 95\% CIs for
each comparison of interest (see table XX in the appendix for full model
output).

\emph{Accuracy.} Next, we modeled accuracy using a mixed-effects
logistic regression with the same specifications (see Panel B of Figur
5). We found that adults' first shifts were highly accurate
(\(M_{bullseye}\) = 0.92, \(M_{face}\) = 0.95, \(M_{text}\) = 0.90,
\(M_{textNoAudio}\) = 0.89). And, in contrast to the children in
Experiment 1, adults' responses were above chance level even in the
Bullseye condition when the center stimulus was not salient or
informative.

Adults' accurate first shifts suggests an interesting developmental
difference in the construal of the center stimulus in our task. This is
speculative, but it seems plausible that adults thought the Bullseye was
designed to be a valid starting point for fixating gaze while the
sentence unfolded (i.e., someone put this here for a reason). As a
result, if adults maintained their fixation on the center stimulus for
enough time to gather sufficient linguistic singal, then they were
highly accurate across all four processing condition, which is
reasonable since these were highly familiar words presented in
child-directed speech.

Visual inspection of the timecourse looking curves, however, suggests
that the effect of the text manipulations occurred earlier in timecourse
of decisions about visual fixation. That is, in the first 300 ms after
the start of the target word, adults in the Bullseye, Face, and Text
conditions, where they had access to linguistic information via the
auditory channel, were already allocating fixations away from the center
stimulus and to the objects. In contrast, in the Text-No-Audio
condition, all of adults' fixation were directed to the center stimulus
location, which contained the language-relevant information. Next, we
use our model-based analyses to quantify these differences in adults'
decisions about where to fixate as a function of time.

\hypertarget{model-based-analyses-1}{%
\subsubsection{Model-based analyses}\label{model-based-analyses-1}}

\begin{figure}[!t]

{\centering \includegraphics[width=0.9\linewidth]{figs/text-model-plots-1} 

}

\caption{Results for the model-based analyses of Experiment 2. All plotting conventions are the same as Figure 3.}\label{fig:text-model-plots}
\end{figure}

\emph{EWMA.} For all four conditions, the control statistic crossed the
upper control limit (see Panel A of Figure 6), suggesting that at some
point in the RT distribution adults' shifts were reliably driven by
linguistic information. Interestingly, we found a graded effect of
condition on the cutoff point (see the shift in the vertical dashed
lines in Panel A of Figure 5). That is, the CS crossed the UCL earliest
in the Text-no-audio condition (\(M_{text-no-audio}\) = 0.39, 95\% HDI
{[}0.37, 0.41{]}), followed by the Text (\(M_{text}\) = 0.44, 95\% HDI
{[}0.42, 0.46{]}) and Face (\(M_{face}\) = 0.45, 95\% HDI {[}0.43,
0.47{]}) conditions, and finally the Bullseye condition
(\(M_{bullseye}\) = 0.54, 95\% HDI {[}0.52, 0.56{]}).\footnote{See Table
  XX in the appendix for the relevant statistics for the pairwise
  comparisons of interest.}

We also found a smiliar pattern of a graded difference in the proportion
of shifts that occurred when the control statistic was below the upper
control limit (\(M_{bullseye}\) = 0.78, \(M_{text}\) = 0.86,
\(M_{text-no-audio}\) = 0.89, \(M_{face}\) = 0.93). Adults generated
fewer language-driven eye movements in the Bullseye condition comapred
to the other contexts (\(\beta\) = -0.12, 95\% HDI {[}-0.26, -0.01{]}).
And the highest proportion of language-driven shifts in the
Text-no-audio context (\(\beta\) = 0.04, 95\% HDI {[}-0.02, 0.08{]}).
These results provide evidence for our key prediction: that increasing
the value of fixating the center stimulus for gathering linguistic
information reduced gaze shifts to the rest of the visual world. This
shift in gaze dynamics, in turn, resulted in adults gathering more of
the linguistic signal prior to generating eye movements away from the
center stimulus, leading to a higher proporiton of language-driven
shifts earlier in the distribution of reaction times.

\emph{HDDM.} Using the classifications generated by the EWMA, we fit a
HDDM to the language-drven shifts with the same specifications as in
Experiment 1. There was high overlap of the posterior distributions for
the drift rate parameters (see panel C of Figure 5), suggesting that
participants gathered the linguistic information with similar
efficiency. We also found high overlap in the distribution of boundary
separation estimates for the Bullseye, Text, and Text-no-audio
conditions. We saw some evidence for a higher boundary separation in the
Face condition compared to the other three center stimulus types (Face
boundary = 1.73, HDI = {[}1.49, 1.98{]}; Bullseye boundary = 1.40, HDI =
{[}1.19, 1.62{]}; Text boundary = 1.37, HDI = {[}1.16, 1.58{]};
Text-no-audio boundary = 1.34, HDI = {[}1.14, 1.55{]}), indicating that
adults' higher accuracy in this condition was driven by accumulating
more information before generating a response. Note that the higher
boundary separation and drift rate parameters for the Face condition
differs from the results of the standard Accuracy analyses, which found
similar patterns of performance. This occurs because the HDDM estimates
parameter fits using reaction times distributions for both correct and
incorrect responses.

\emph{Results summary.} Together, these results suggest that adults were
sensitive to the tradeoff between gathering different kinds of visual
information. When processing text, people generated fewer
nonlanguage-driven shifts (EWMA results) but their processing efficiency
of the linguistic signal itself did not change (HDDM results).
Interestingly, we found a graded difference in the EWMA results between
the Text and Text-no-audio conditions, with the lowest proportion of
early, nonlanguage-driven shifts occurring while processing text without
the verbal stimuli. This behavior makes sense; if the adults could rely
on the auditory channel to gather the linguistic information, then the
value of fixating the text display decreases. In contrast to the
children in Experiment 1, adults were highly accurate in the Bullseye
condition, perhaps because they construed the Bullseye as a center
fixation that they \emph{should} fixate, or perhaps they had better
encoded the location/identity of the two referents prior to the start of
the target sentence.

\hypertarget{experiment-3}{%
\section{Experiment 3}\label{experiment-3}}

In this experiment, we recorded adults and children's eye movements
during a real-time language comprehension task where participants
processed familiar sentences (e.g., \enquote{Where's the ball?}) while
looking at a simplified visual world with three fixation targets. Using
a within-participants design, we manipulated the signal-to-noise ratio
of the auditory signal by convolving the acoustic input with brown noise
(random noise with greater energy at lower frequencies).

We predicted that processing speech in a noisy context would make
participants less likely to shift before collecting sufficient
information.\footnote{See \url{https://osf.io/g8h9r/} for a
  pre-registration of the analysis plan.} This delay, in turn, would
lead to a lower proportion of shifts flagged as random/exploratory in
the EWMA analysis, and a pattern of DDM results indicating a
prioritization of accuracy over and above speed of responding (see the
Analysis Plan section below for more details on the models). We also
predicted a developmental difference -- that children would produce a
higher proportion of random shifts and accumulate information less
efficiently compared to adults; and a developmental parallel -- that
children would show the same pattern of adapting gaze patterns to gather
more visual information in the noisy processing context.

\hypertarget{methods-2}{%
\subsection{Methods}\label{methods-2}}

\hypertarget{participants-2}{%
\subsubsection{Participants}\label{participants-2}}

Participants were native, monolingual English-learning children (\(n=\)
39; 22 F) and adults (\(n=\) 31; 22 F). All participants had no reported
history of developmental or language delay and normal vision. 14
participants (11 children, 3 adults) were run but not included in the
analysis because either the eye tracker falied to calibrate (2 children,
3 adults) or the participant did not complete the task (9 children).

\hypertarget{stimuli-2}{%
\subsubsection{Stimuli}\label{stimuli-2}}

\emph{Linguistic stimuli.} The video/audio stimuli were recorded in a
sound-proof room and featured two female speakers who used natural
child-directed speech and said one of two phrases: \enquote{Hey! Can you
find the (target word)} or "Look! Where's the (target word) -- see panel
A of Fig.~\ref{fig:stimuli_plot}. The target words were: ball, bunny,
boat, bottle, cookie, juice, chicken, and shoe. The target words varied
in length (shortest = 411.68 ms, longest = 779.62 ms) with an average
length of 586.71 ms.

\emph{Noise manipulation}. To create the stimuli in the noise condition,
we convolved each recording with Brown noise using the Audacity audio
editor. The average signal-to-noise ratio\footnote{The ratio of signal
  power to the noise power, with values greater than 0 dB indicating
  more signal than noise.} in the noise condition was 2.87 dB compared
to the clear condition, which was 35.05 dB.

\emph{Visual stimuli.} The image set consisted of colorful digitized
pictures of objects presented in fixed pairs with no phonological
overlap between the target and the distractor image (cookie-bottle,
boat-juice, bunny-chicken, shoe-ball). The side of the target picture
was counterbalanced across trials.

\hypertarget{design-and-procedure-2}{%
\subsubsection{Design and procedure}\label{design-and-procedure-2}}

Participants viewed the task on a screen while their gaze was tracked
using an SMI RED corneal-reflection eye-tracker mounted on an LCD
monitor, sampling at 60 Hz. The eye-tracker was first calibrated for
each participant using a 6-point calibration. On each trial,
participants saw two images of familiar objects on the screen for two
seconds before the center stimulus appeared (see
Fig.~\ref{fig:stimuli_plot}). Next, they processed the target sentence
-- which consisted of a carrier phrase, a target noun, and a question --
followed by two seconds without language to allow for a response. Child
participants saw 32 trials (16 noise trials; 16 clear trials) with
several filler trials interspersed to maintain interest. Adult
participants saw 64 trials (32 noise; 32 clear). The noise manipulation
was presented in a blocked design with the order of block
counterbalanced across participants.

\hypertarget{results-and-discussion-2}{%
\subsection{Results and discussion}\label{results-and-discussion-2}}

\hypertarget{behavioral-analyses-2}{%
\subsubsection{Behavioral analyses:}\label{behavioral-analyses-2}}

\begin{figure}[!t]

{\centering \includegraphics[width=0.85\linewidth]{/Users/kmacdonald/Documents/Projects/SPEED-ACC/paper/journal_submission/figures/figs_output//fig5_noise_behav} 

}

\caption{Behavioral results for children and adults in Experiment 3. Panel A shows the overall looking to the center, target, and distracter stimulus for each processing condition and age group. Panel B shows the distribution of RTs for each participant and the pairwise contrast between the noise and clear conditions. The square point represents the mean value for each mesure. The vertical dashed line represents the null model of zero condition difference. The width each point represents the 95\% HDI. Panel C shows the same information but for first shift accuracy.}\label{fig:noise-acc-rt}
\end{figure}

\textbf{RT.} To make RTs more suitable for modeling on a linear scale,
we analyzed responses in log space with the final model specified as:
\texttt{$log(RT) \sim noise\_condition + age\_group + (noise\_condition \mid sub\_id ) + (noise\_condition \mid target\_item)$}.
Panel A of Figure ~\ref{fig:noise_acc_rt_noise_plot} shows the full RT
data distribution and the full posterior distribution of the estimated
RT difference between the noise and clear conditions. Both children and
adults were slower to identify the target in the noise condition
(Children \(M_{noise}\) = 500.19 sec; Adult \(M_{noise}\) = 595.23 sec),
as compared to the clear condition (Children \(M_{clear}\) = 455.72 sec
Adult \(M_{clear}\) = 542.45 sec). RTs in the noise condition were 48.82
seconds slower on average, with a 95\% HDI ranging from 3.72 sec to
96.26 ms, and not including the null value of zero condition difference.
Older children responded faster than younger children (\(M_{age}\) =
-0.44, {[}-0.74, -0.16{]}), with little evidence for an interaction
between age and condition.

\textbf{Accuracy.} Next, we modeled adults and children's first shift
accuracy using a mixed-effects logistic regression with the same
specifications (see Panel B of Fig.~\ref{fig:noise_acc_rt_noise_plot}).
Both groups were more accurate than a model of random responding (null
value of \(0.5\) falling well outside the lower bound of the 95\% HDI
for all group means). Adults were more accurate (\(M_{adults} =\) 90\%)
than children (\(M_{children} =\) 61\%). The key result is that both
groups showed evidence of higher accuracy in the noise condition:
children (\(M_{noise}\) = 67\%; \(M_{clear}\) = 61\%) and adults
(\(M_{noise}\) = 92\%; \(M_{clear}\) = 90\%). Accuracy in the noise
condition was on average 4\% higher, with a 95\% HDI from -1\% to 12\%.
Note that the null value of zero difference falls at the very edge of
the HDI. But 95\% of the credible values are greater than zero,
providing evidence for higher accuracy in the noise condition. Within
the child sample, there was no evidence of a main effect of age or an
interaction between age and noise condition.

\hypertarget{model-based-analyses-2}{%
\subsubsection{Model-based analyses:}\label{model-based-analyses-2}}

\begin{figure}[!t]

{\centering \includegraphics[width=0.85\linewidth]{/Users/kmacdonald/Documents/Projects/SPEED-ACC/paper/journal_submission/figures/figs_output//fig6_noise_models} 

}

\caption{Results for the model-based analyses for Experiment 3. The majority of plotting conventions are the same as Figure 3. In Panel C, linetype and alpha value represent age group: children vs. adults.}\label{fig:noise-model-plots}
\end{figure}

\textbf{EWMA.} Fig.~\ref{fig:noise_ewma_violin_plot} shows the
proportion of shifts that the model classified as random
vs.~language-driven for each age group and processing context. On
average, 41\% (95\% HDI: 32\%, 50\%) of children's shifts were
categorized as language-driven, which was significantly fewer than
adults, 87\% (95\% HDI: 78\%, 96\%). Critically, processing speech in a
noisy context caused both adults and children to generate a higher
proportion of language-driven shifts (i.e., fewer random, exploratory
shifts away from the speaker), with the 95\% HDI excluding the null
value of zero condition difference (\(\beta_{noise}\) = 11\%, {[}7.00\%,
16\%{]}). Within the child sample, older children generated fewer
random, early shifts (\(M_{age}\) = -0.21, {[}-0.35, -0.08{]}). There
was no eivdence of an interaction between age and condition. This
pattern of results suggests that the noise condition caused participants
to increase visual fixations to the language source, leading them to
generate fewer exploratory, random shifts before accumulating sufficient
information to respond accurately.

\textbf{HDDM.} Fig.~\ref{fig:hddm_plot_noise} shows the full posterior
distributions for the HDDM output. Children had lower drift rates
(children \(M_{drift}\) = NA; adults \(M_{drift}\) = NA) and boundary
separation estimates (children \(M_{boundary}\) = 1.02; adults
\(M_{boundary}\) = 1.33) as compared to adults, suggesting that children
were less efficient and less cautious in their responding. The noise
manipulation selectively affected the boundary separation parameter,
with higher estimates in the noise condition for both age groups
(\(\beta_{noise}\) = 0.26, {[}0.10, 0.42{]}). This result suggests that
participants' in the noise condition prioritized information
accumulation over speed when generating an eye movement in response to
the incoming language. This increased decision threshold led to higher
accuracy. Moreover, the high overlap in estimates of drift rate suggests
that participants were able to integrate the visual and auditory signals
such that they could achieve a level of processing efficiency comparable
to the clear processing context.

Taken together, the behavioral and EWMA/HDDM results provide converging
support for the predictions of our information-seeking account.
Processing speech in noise caused listeners to seek additional visual
information to support language comprehension. Moreover, we observed a
very similar pattern of behavior in children and adults, with both
groups producing more language-driven shifts and prioritizing accuracy
over speed in the more challenging, noisy environment.

\hypertarget{general-discussion}{%
\section{General Discussion}\label{general-discussion}}

Language comprehension in grounded, social contexts involves extracting
meaning from the linguistic signal and mapping it to the surrounding
world. But how should listeners prioritize integrating different kinds
of visual information? In this work, we proposed that listeners flexibly
adapt their gaze behaviors in response to features of the social
context, seeking visual information from their social partners when it
was especially useful for language comprehension. We present evidence
for our account using three case studies sampled from a diverse set of
language processing contexts: sign language, dynamic displays of printed
text, and spoken language in noisy auditory environments. We found that,
compared to children learning spoken English, young ASL-learners delayed
their gaze shifts away from a language source, were more accurate with
these shifts, and produced a smaller proportion of random shifting
behavior. Next, English-speaking adults produced fewer random gaze
shifts when processing dynamic displays of printed text compared to
spoken language. Finally, 3-5 year-olds and adults delayed the timing of
gaze shifts away from a speaker's face when processing speech in a noisy
environment, which resulted in fewer random eye movements and more
accurate gaze shifts. Together, these results provide evidence that
young listeners, like adults, adapt their gaze patterns to the demands
of different processing environments by seeking out visual information
from social partners to support language comprehension.

This work attempts to integrate top-down, goal-based models of vision
(Hayhoe \& Ballard, 2005) with work on language-driven eye movements
(Allopenna et al., 1998). While we chose to start with two case studies
-- ASL and text processing -- we think the account is more general and
that there are many real world situations where people must negotiate
the tradeoff between gathering more information about language or about
the world: e.g., processing spoken language in noisy environments or at
a distance; or early in language learning when children are acquiring
new words and often rely on nonlinguistic cues to reference such as
pointing or eye gaze. Overall, we hope this work contributes to a
broader account of eye movements during language comprehension that can
explain fixation behaviors across a wider variety of populations,
processing contexts, and during different stages of language learning.

These results synthesize ideas from several research programs, including
work on language-mediated visual attention (Tanenhaus et al., 1995),
goal-based accounts of vision during everyday tasks (Hayhoe \& Ballard,
2005), and work on effortful listening (Van Engen \& Peelle, 2014).
Moreover, our findings parallel recent work by McMurray, Farris-Trimble,
and Rigler (2017) showing that individuals with Cochlear Implants, who
are consistently processing degraded auditory input, are more likely to
delay the process of lexical access as measured by slower gaze shifts to
named referents and fewer incorrect gaze shifts to phonological onset
competitors. McMurray et al. (2017) also found that they could replicate
these changes to gaze patterns in adults with typical hearing by
degrading the auditory stimuli so that it shared features with the
output of a cochlear implant (noise-vocoded speech).

The results reported here also dovetail with recent developmental work
by Yurovsky et al. (2017). In that study, preschoolers, like adults,
were able to integrate top-down expectations about the kinds of things
speakers are likely to talk about with bottom-up cues from auditory
perception. Yurovsky et al. (2017) situated this finding within the
framework of modeling language as a \emph{noisy channel} where listeners
combine expectations with perceptual data and weight each based on its
reliability. Here, we found a similar developmental parallel in language
processing: that 3-5 year-olds, like adults, adapted their gaze patterns
to seek additional visual information when the auditory signal became
less reliable. This adaptation allowed listeners to generate more
accurate responses in the more challenging, noisy context.

\hypertarget{limitations}{%
\subsection{Limitations}\label{limitations}}

This work has several important limitations that pave the way for future
work. First, we chose to focus on a single decision about visual
fixation to provide a window onto the dynamics of decision-making across
different language processing contexts. But our analysis does not
consider the rich information present in the gaze patterns that occur
leading up to this decision. In our future work, we aim to measure how
changes in the language environment might lead to shifts in the dynamics
of gaze across a wider timescale. For example, perhaps listeners gather
more information about the objects in the scene before the sentence in
anticipation of allocating more attention to the speaker once they start
to speak. Second, we chose one instantiation of a noisy processing
context -- random background noise. But we think our findings should
generalize to contexts where other kinds of noise -- e.g., uncertainty
over a speaker's reliability or when processing accented speech -- make
gathering visual information from the speaker more useful for language
understanding.

\hypertarget{conclusion}{%
\subsection{Conclusion}\label{conclusion}}

This experiment tested the generalizability of our information-seeking
account of eye movements within the domain of grounded language
comprehension. But the account could be applied to the language
acquisition context. Consider that early in language learning, children
are acquiring novel word-object links while also learning about visual
object categories. Both of these tasks produce different goals that
should, in turn, modulate children's decisions about where to allocate
visual attention -- e.g., seeking nonlinguistic cues to reference such
as eye gaze and pointing become critical when you are unfamiliar with
the information in the linguistic signal. More generally, this work
integrates goal-based models of eye-movements with language
comprehension in grounded, social contexts. This approach presents a way
forward for explaining fixation behaviors across a wider variety
processing contexts and during different stages of language learning.

\newpage

\hypertarget{references}{%
\section{References}\label{references}}

\setlength{\parindent}{-0.5in}
\setlength{\leftskip}{0.5in}

\hypertarget{refs}{}
\leavevmode\hypertarget{ref-allopenna1998tracking}{}%
Allopenna, P. D., Magnuson, J. S., \& Tanenhaus, M. K. (1998). Tracking
the time course of spoken word recognition using eye movements: Evidence
for continuous mapping models. \emph{Journal of Memory and Language},
\emph{38}(4), 419--439.

\leavevmode\hypertarget{ref-altmann2007real}{}%
Altmann, G., \& Kamide, Y. (2007). The real-time mediation of visual
attention by language and world knowledge: Linking anticipatory (and
other) eye movements to linguistic processing. \emph{Journal of Memory
and Language}, \emph{57}(4), 502--518.

\leavevmode\hypertarget{ref-byers2017bilingual}{}%
Byers-Heinlein, K., Morin-Lessard, E., \& Lew-Williams, C. (2017).
Bilingual infants control their languages as they listen.
\emph{Proceedings of the National Academy of Sciences}, \emph{114}(34),
9032--9037.

\leavevmode\hypertarget{ref-dahan2005looking}{}%
Dahan, D., \& Tanenhaus, M. K. (2005). Looking at the rope when looking
for the snake: Conceptually mediated eye movements during spoken-word
recognition. \emph{Psychonomic Bulletin \& Review}, \emph{12}(3),
453--459.

\leavevmode\hypertarget{ref-erber1969interaction}{}%
Erber, N. P. (1969). Interaction of audition and vision in the
recognition of oral speech stimuli. \emph{Journal of Speech and Hearing
Research}, \emph{12}(2), 423--425.

\leavevmode\hypertarget{ref-estigarribia2007getting}{}%
Estigarribia, B., \& Clark, E. V. (2007). Getting and maintaining
attention in talk to young children. \emph{Journal of Child Language},
\emph{34}(4), 799--814.

\leavevmode\hypertarget{ref-fernald2006picking}{}%
Fernald, A., Perfors, A., \& Marchman, V. A. (2006). Picking up speed in
understanding: Speech processing efficiency and vocabulary growth across
the 2nd year. \emph{Developmental Psychology}, \emph{42}(1), 98.

\leavevmode\hypertarget{ref-fernald2008looking}{}%
Fernald, A., Zangl, R., Portillo, A. L., \& Marchman, V. A. (2008).
Looking while listening: Using eye movements to monitor spoken language.
\emph{Developmental Psycholinguistics: On-Line Methods in Children's
Language Processing}, \emph{44}, 97.

\leavevmode\hypertarget{ref-gabry2016rstanarm}{}%
Gabry, J., \& Goodrich, B. (2016). Rstanarm: Bayesian applied regression
modeling via stan. R package version 2.10. 0.

\leavevmode\hypertarget{ref-gold2000representation}{}%
Gold, J. I., \& Shadlen, M. N. (2000). Representation of a perceptual
decision in developing oculomotor commands. \emph{Nature},
\emph{404}(6776), 390.

\leavevmode\hypertarget{ref-hayhoe2005eye}{}%
Hayhoe, M., \& Ballard, D. (2005). Eye movements in natural behavior.
\emph{Trends in Cognitive Sciences}, \emph{9}(4), 188--194.

\leavevmode\hypertarget{ref-hoppe2016learning}{}%
Hoppe, D., \& Rothkopf, C. A. (2016). Learning rational temporal eye
movement strategies. \emph{Proceedings of the National Academy of
Sciences}, \emph{113}(29), 8332--8337.

\leavevmode\hypertarget{ref-huettig2005word}{}%
Huettig, F., \& Altmann, G. T. (2005). Word meaning and the control of
eye fixation: Semantic competitor effects and the visual world paradigm.
\emph{Cognition}, \emph{96}(1), B23--B32.

\leavevmode\hypertarget{ref-kelly2010two}{}%
Kelly, S. D., Özyürek, A., \& Maris, E. (2010). Two sides of the same
coin: Speech and gesture mutually interact to enhance comprehension.
\emph{Psychological Science}, \emph{21}(2), 260--267.

\leavevmode\hypertarget{ref-liszkowski2012prelinguistic}{}%
Liszkowski, U., Brown, P., Callaghan, T., Takada, A., \& De Vos, C.
(2012). A prelinguistic gestural universal of human communication.
\emph{Cognitive Science}, \emph{36}(4), 698--713.

\leavevmode\hypertarget{ref-macdonald1978visual}{}%
MacDonald, J., \& McGurk, H. (1978). Visual influences on speech
perception processes. \emph{Attention, Perception, \& Psychophysics},
\emph{24}(3), 253--257.

\leavevmode\hypertarget{ref-macdonald2018real}{}%
MacDonald, K., LaMarr, T., Corina, D., Marchman, V. A., \& Fernald, A.
(2018). Real-time lexical comprehension in young children learning
american sign language. \emph{Developmental Science}, e12672.

\leavevmode\hypertarget{ref-macdonald2006constraint}{}%
MacDonald, M. C., \& Seidenberg, M. S. (2006). Constraint satisfaction
accounts of lexical and sentence comprehension. \emph{Handbook of
Psycholinguistics}, \emph{2}, 581--611.

\leavevmode\hypertarget{ref-marchman2008speed}{}%
Marchman, V. A., \& Fernald, A. (2008). Speed of word recognition and
vocabulary knowledge in infancy predict cognitive and language outcomes
in later childhood. \emph{Developmental Science}, \emph{11}(3).

\leavevmode\hypertarget{ref-mcclelland1986trace}{}%
McClelland, J. L., \& Elman, J. L. (1986). The trace model of speech
perception. \emph{Cognitive Psychology}, \emph{18}(1), 1--86.

\leavevmode\hypertarget{ref-mcclelland2006there}{}%
McClelland, J. L., Mirman, D., \& Holt, L. L. (2006). Are there
interactive processes in speech perception? \emph{Trends in Cognitive
Sciences}, \emph{10}(8), 363--369.

\leavevmode\hypertarget{ref-mcmurray2017waiting}{}%
McMurray, B., Farris-Trimble, A., \& Rigler, H. (2017). Waiting for
lexical access: Cochlear implants or severely degraded input lead
listeners to process speech less incrementally. \emph{Cognition},
\emph{169}, 147--164.

\leavevmode\hypertarget{ref-nelson2007probabilistic}{}%
Nelson, J. D., \& Cottrell, G. W. (2007). A probabilistic model of eye
movements in concept formation. \emph{Neurocomputing}, \emph{70}(13-15),
2256--2272.

\leavevmode\hypertarget{ref-peelle2015prediction}{}%
Peelle, J. E., \& Sommers, M. S. (2015). Prediction and constraint in
audiovisual speech perception. \emph{Cortex}, \emph{68}, 169--181.

\leavevmode\hypertarget{ref-ratcliff2015individual}{}%
Ratcliff, R., \& Childers, R. (2015). Individual differences and fitting
methods for the two-choice diffusion model of decision making.
\emph{Decision}, \emph{2}(4), 237--279.

\leavevmode\hypertarget{ref-rigler2015slow}{}%
Rigler, H., Farris-Trimble, A., Greiner, L., Walker, J., Tomblin, J. B.,
\& McMurray, B. (2015). The slow developmental time course of real-time
spoken word recognition. \emph{Developmental Psychology}, \emph{51}(12),
1690.

\leavevmode\hypertarget{ref-salverda2011goal}{}%
Salverda, A. P., Brown, M., \& Tanenhaus, M. K. (2011). A goal-based
perspective on eye movements in visual world studies. \emph{Acta
Psychologica}, \emph{137}(2), 172--180.

\leavevmode\hypertarget{ref-schwartz2013language}{}%
Schwartz, R. G., Steinman, S., Ying, E., Mystal, E. Y., \& Houston, D.
M. (2013). Language processing in children with cochlear implants: A
preliminary report on lexical access for production and comprehension.
\emph{Clinical Linguistics \& Phonetics}, \emph{27}(4), 264--277.

\leavevmode\hypertarget{ref-shinoda2001controls}{}%
Shinoda, H., Hayhoe, M. M., \& Shrivastava, A. (2001). What controls
attention in natural environments? \emph{Vision Research},
\emph{41}(25-26), 3535--3545.

\leavevmode\hypertarget{ref-spivey2002eye}{}%
Spivey, M. J., Tanenhaus, M. K., Eberhard, K. M., \& Sedivy, J. C.
(2002). Eye movements and spoken language comprehension: Effects of
visual context on syntactic ambiguity resolution. \emph{Cognitive
Psychology}, \emph{45}(4), 447--481.

\leavevmode\hypertarget{ref-tanenhaus1995integration}{}%
Tanenhaus, M. K., Spivey-Knowlton, M. J., Eberhard, K. M., \& Sedivy, J.
C. (1995). Integration of visual and linguistic information in spoken
language comprehension. \emph{Science}, \emph{268}(5217), 1632.

\leavevmode\hypertarget{ref-vandekerckhove2007fitting}{}%
Vandekerckhove, J., \& Tuerlinckx, F. (2007). Fitting the ratcliff
diffusion model to experimental data. \emph{Psychonomic Bulletin \&
Review}, \emph{14}(6), 1011--1026.

\leavevmode\hypertarget{ref-van2014listening}{}%
Van Engen, K. J., \& Peelle, J. E. (2014). Listening effort and accented
speech. \emph{Frontiers in Human Neuroscience}, \emph{8}.

\leavevmode\hypertarget{ref-venker2013individual}{}%
Venker, C. E., Eernisse, E. R., Saffran, J. R., \& Weismer, S. E.
(2013). Individual differences in the real-time comprehension of
children with asd. \emph{Autism Research}, \emph{6}(5), 417--432.

\leavevmode\hypertarget{ref-vigliocco2014language}{}%
Vigliocco, G., Perniss, P., \& Vinson, D. (2014). Language as a
multimodal phenomenon: Implications for language learning, processing
and evolution. The Royal Society.

\leavevmode\hypertarget{ref-wiecki2013hddm}{}%
Wiecki, T. V., Sofer, I., \& Frank, M. J. (2013). HDDM: Hierarchical
bayesian estimation of the drift-diffusion model in python.
\emph{Frontiers in Neuroinformatics}, \emph{7}, 14.

\leavevmode\hypertarget{ref-yee2006eye}{}%
Yee, E., \& Sedivy, J. C. (2006). Eye movements to pictures reveal
transient semantic activation during spoken word recognition.
\emph{Journal of Experimental Psychology: Learning, Memory, and
Cognition}, \emph{32}(1), 1.

\leavevmode\hypertarget{ref-yurovsky2017preschoolers}{}%
Yurovsky, D., Case, S., \& Frank, M. C. (2017). Preschoolers flexibly
adapt to linguistic input in a noisy channel. \emph{Psychological
Science}, \emph{28}(1), 132--140.






\end{document}
