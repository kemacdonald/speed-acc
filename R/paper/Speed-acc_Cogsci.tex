% Template for Cogsci submission with R Markdown

% Stuff changed from original Markdown PLOS Template
\documentclass[10pt, letterpaper]{article}

\usepackage{cogsci}
\usepackage{pslatex}
\usepackage{float}
\usepackage{caption}

% amsmath package, useful for mathematical formulas
\usepackage{amsmath}

% amssymb package, useful for mathematical symbols
\usepackage{amssymb}

% hyperref package, useful for hyperlinks
\usepackage{hyperref}

% graphicx package, useful for including eps and pdf graphics
% include graphics with the command \includegraphics
\usepackage{graphicx}

% Sweave(-like)
\usepackage{fancyvrb}
\DefineVerbatimEnvironment{Sinput}{Verbatim}{fontshape=sl}
\DefineVerbatimEnvironment{Soutput}{Verbatim}{}
\DefineVerbatimEnvironment{Scode}{Verbatim}{fontshape=sl}
\newenvironment{Schunk}{}{}
\DefineVerbatimEnvironment{Code}{Verbatim}{}
\DefineVerbatimEnvironment{CodeInput}{Verbatim}{fontshape=sl}
\DefineVerbatimEnvironment{CodeOutput}{Verbatim}{}
\newenvironment{CodeChunk}{}{}

% cite package, to clean up citations in the main text. Do not remove.
\usepackage{cite}

\usepackage{color}

% Use doublespacing - comment out for single spacing
%\usepackage{setspace}
%\doublespacing


% % Text layout
% \topmargin 0.0cm
% \oddsidemargin 0.5cm
% \evensidemargin 0.5cm
% \textwidth 16cm
% \textheight 21cm

\title{Comparing the real-time dynamics of language-mediated eye movements in
American Sign Language and spoken English}


\author{{\large \bf Kyle MacDonald} \\ \texttt{kylem4@stanford.edu} \\ Department of Psychology \\ Stanford University \And {\large \bf Virginia Marchman} \\ \texttt{marchman@stanford.edu} \\ Department of Psychology \\ Stanford University
    \And {\large \bf Anne Fernald} \\ \texttt{afernald@stanford.edu} \\ Department of Psychology \\ Stanford University
\And {\large \bf Michael C. Frank} \\ \texttt{mcfrank@stanford.edu} \\ Department of Psychology \\ Stanford University}

\begin{document}

\maketitle

\begin{abstract}
The abstract should be one paragraph, indented 1/8 inch on both sides,
in 9 point font with single spacing. The heading Abstract should be 10
point, bold, centered, with one line space below it. This one-paragraph
abstract section is required only for standard spoken papers and
standard posters (i.e., those presentations that will be represented by
six page papers in the Proceedings).

\textbf{Keywords:}
eye movements; language processing; American Sign Language; sequential
decision making models
\end{abstract}

\section{Introduction}\label{introduction}

\section{Experiment 1}\label{experiment-1}

\subsection{Method}\label{method}

\subsection{Results}\label{results}

\subsubsection{Behavioral analyses}\label{behavioral-analyses}

\subsubsection{Model-based analyses}\label{model-based-analyses}

\subsection{Discussion}\label{discussion}

\section{Experiment 2}\label{experiment-2}

\subsection{Method}\label{method-1}

\subsection{Results}\label{results-1}

\subsubsection{Behavioral analyses}\label{behavioral-analyses-1}

\subsubsection{Model-based analyses}\label{model-based-analyses-1}

\subsection{Discussion}\label{discussion-1}

\section{General Discussion}\label{general-discussion}

\section{Acknowledgements}\label{acknowledgements}

We are grateful to Aviva Blonder for help with data collection and
analysis. We thank the members of the Language and Cognition Lab and the
Social Learning Lab for their helpful feedback on this project. This
work was supported by a National Science Foundation Graduate Research
Fellowship to KM.

\section{References}\label{references}

\setlength{\parindent}{-0.1in} \setlength{\leftskip}{0.125in} \noindent

\end{document}
