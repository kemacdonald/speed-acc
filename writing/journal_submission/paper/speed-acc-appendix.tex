\clearpage
\makeatletter
\efloat@restorefloats
\makeatother


\begin{appendix}
\hypertarget{parallel-set-of-non-bayesian-analyses}{%
\section{Parallel set of non-Bayesian
analyses}\label{parallel-set-of-non-bayesian-analyses}}

In this appendix, we report the output of a parallel set of non-Bayesian
linear mixed-effects models for the primary analyses (Reaction Time,
Accuracy, EWMA) in Experiments 1 and 2. The \(p\) values for each
coefficient were computed via the Satterthwaite approximation using the
\texttt{lmerTest} R package (Kuznetsova, Brockhoff, \& Christensen,
2017).

\begin{table}[h]
\begin{center}
\begin{threeparttable}
\caption{\label{tab:mle rt trio}Output of the MLE model predicting log reaction time as a function of center stimulus type in Experiment 1.}
\begin{tabular}{llllll}
\toprule
term & \multicolumn{1}{c}{estimate} & \multicolumn{1}{c}{std.error} & \multicolumn{1}{c}{statistic} & \multicolumn{1}{c}{df} & \multicolumn{1}{c}{p.value}\\
\midrule
Intercept (ASL) & 6.96 & 0.08 & 91.84 & 62.58 & < .001\\
Face & -0.37 & 0.11 & -3.32 & 84.45 & < .01\\
Object & -1.12 & 0.10 & -11.13 & 79.76 & < .001\\
Bullseye & -1.30 & 0.13 & -10.04 & 97.77 & < .001\\
scale(Months) & -0.01 & 0.04 & -0.18 & 75.55 & 0.86\\
\bottomrule
\addlinespace
\end{tabular}
\begin{tablenotes}[para]
\normalsize{\textit{Note.} Model specification: log(RT) \textasciitilde{} stimuli + scale(Months) + (1 | subid) + (1 | target\_image)}
\end{tablenotes}
\end{threeparttable}
\end{center}
\end{table}

\begin{table}[h]
\begin{center}
\begin{threeparttable}
\caption{\label{tab:mle acc trio}Output of the MLE model predicting accuracy as a function of center stimulus type in Experiment 1.}
\begin{tabular}{lllll}
\toprule
term & \multicolumn{1}{c}{estimate} & \multicolumn{1}{c}{std.error} & \multicolumn{1}{c}{statistic} & \multicolumn{1}{c}{p.value}\\
\midrule
Intercept (ASL) & 1.07 & 0.14 & 7.66 & < .001\\
Face & -0.64 & 0.20 & -3.19 & < .01\\
Object & -1.14 & 0.19 & -6.16 & < .001\\
Bullseye & -1.15 & 0.22 & -5.26 & < .001\\
\bottomrule
\addlinespace
\end{tabular}
\begin{tablenotes}[para]
\normalsize{\textit{Note.} Model specification: correct \textasciitilde{} stimuli + (1 | subid) + (1 | target\_image)}
\end{tablenotes}
\end{threeparttable}
\end{center}
\end{table}

\begin{table}[h]
\begin{center}
\begin{threeparttable}
\caption{\label{tab:mle ewma trio prop lang}Output of the MLE model estimating the difference in random responding in the EWMA as a function of center stimulus type in Experiment 1.}
\begin{tabular}{lllll}
\toprule
term & \multicolumn{1}{c}{estimate} & \multicolumn{1}{c}{std.error} & \multicolumn{1}{c}{statistic} & \multicolumn{1}{c}{p.value}\\
\midrule
Intercept (ASL) & 1.73 & 0.21 & 8.31 & < .001\\
Face & -1.80 & 0.29 & -6.26 & < .001\\
\bottomrule
\addlinespace
\end{tabular}
\begin{tablenotes}[para]
\normalsize{\textit{Note.} Model specification: guess \textasciitilde{} stimuli + (1 | subid) + (1 | target\_image)}
\end{tablenotes}
\end{threeparttable}
\end{center}
\end{table}

\begin{table}[h]
\begin{center}
\begin{threeparttable}
\caption{\label{tab:mle rt noise}Output of the MLE model predicting log reaction time as a function of noise condition and age category (adults vs. children) in Experiment 2.}
\begin{tabular}{llllll}
\toprule
term & \multicolumn{1}{c}{estimate} & \multicolumn{1}{c}{std.error} & \multicolumn{1}{c}{statistic} & \multicolumn{1}{c}{df} & \multicolumn{1}{c}{p.value}\\
\midrule
Intercept (Clear) & -0.76 & 0.05 & -13.92 & 49.62 & < .001\\
noise & 0.09 & 0.04 & 2.31 & 54.35 & 0.02\\
adults & 0.16 & 0.06 & 2.88 & 56.01 & < .01\\
\bottomrule
\addlinespace
\end{tabular}
\begin{tablenotes}[para]
\normalsize{\textit{Note.} Model specification: log(rt) \textasciitilde{} noise\_condition + age\_category + (noise\_condition | subid) + (1 | target\_image)}
\end{tablenotes}
\end{threeparttable}
\end{center}
\end{table}

\begin{table}[h]
\begin{center}
\begin{threeparttable}
\caption{\label{tab:mle acc noise}Output of the MLE model predicting accuracy as a function of noise condition and age category (adults vs. children) in Experiment 2.}
\begin{tabular}{lllll}
\toprule
term & \multicolumn{1}{c}{estimate} & \multicolumn{1}{c}{std.error} & \multicolumn{1}{c}{statistic} & \multicolumn{1}{c}{p.value}\\
\midrule
Intercept (Clear) & 0.48 & 0.13 & 3.78 & < .001\\
noise & 0.22 & 0.14 & 1.52 & 0.13\\
adults & 1.75 & 0.15 & 11.54 & < .001\\
\bottomrule
\addlinespace
\end{tabular}
\begin{tablenotes}[para]
\normalsize{\textit{Note.} Model specification: correct \textasciitilde{} noise\_condition + age\_category + (noise\_condition | subid) + (1 | target\_image)}
\end{tablenotes}
\end{threeparttable}
\end{center}
\end{table}

\begin{table}[h]
\begin{center}
\begin{threeparttable}
\caption{\label{tab:mle ewma noise}Output of the MLE model estimating the difference in random responding (EWMA model) as a function of noise condition and age category (adults vs. children) in Experiment 2.}
\begin{tabular}{lllll}
\toprule
term & \multicolumn{1}{c}{estimate} & \multicolumn{1}{c}{std.error} & \multicolumn{1}{c}{statistic} & \multicolumn{1}{c}{p.value}\\
\midrule
Intercept (Clear) & 1.81 & 0.15 & 12.23 & < .001\\
noise & 0.73 & 0.17 & 4.18 & < .001\\
children & -2.56 & 0.10 & -25.75 & < .001\\
\bottomrule
\addlinespace
\end{tabular}
\begin{tablenotes}[para]
\normalsize{\textit{Note.} Model specification: guess\_num \textasciitilde{} noise\_condition + age\_category + (noise\_condition | subid) + (noise\_condition | target\_image)}
\end{tablenotes}
\end{threeparttable}
\end{center}
\end{table}

\hypertarget{bayesian-data-analysis-models-experiment-1}{%
\section{Bayesian data analysis models: Experiment
1}\label{bayesian-data-analysis-models-experiment-1}}

In this appendix, we report the full output of the Bayesian linear mixed
effects models in Experiment 1.

\begin{table}[h]
\begin{center}
\begin{threeparttable}
\caption{\label{tab:trio-rt-model}Output of the regression predicting reaction time (milliseconds) as a function of center stimulus type in Experiment 1.}
\begin{tabular}{lll}
\toprule
Center Stimulus Type & \multicolumn{1}{c}{Mean RT} & \multicolumn{1}{c}{95\% HDI}\\
\midrule
Bullseye & 288.71 & [229.8, 355.17]\\
Object & 344.51 & [295.71, 397.31]\\
Face & 724.78 & [603.2, 862.81]\\
ASL & 1,047.89 & [897.29, 1213.52]\\
\bottomrule
\end{tabular}
\end{threeparttable}
\end{center}
\end{table}

\begin{table}[h]
\begin{center}
\begin{threeparttable}
\caption{\label{tab:trio-acc-model}Output of the logistic regression predicting accuracy as a function of center stimulus type in Experiment 1.}
\begin{tabular}{lll}
\toprule
Center Stimulus Type & \multicolumn{1}{c}{Mean Accuracy} & \multicolumn{1}{c}{95\% HDI}\\
\midrule
Object & 0.46 & [0.37, 0.53]\\
Bullseye & 0.48 & [0.38, 0.58]\\
Face & 0.63 & [0.54, 0.7]\\
ASL & 0.76 & [0.7, 0.81]\\
\bottomrule
\end{tabular}
\end{threeparttable}
\end{center}
\end{table}

\begin{table}[h]
\begin{center}
\begin{threeparttable}
\caption{\label{tab:trio-acc-contrasts}Output of the model estimating differences in Accuracy for specific contrasts of interest in Experiment 1.}
\begin{tabular}{lll}
\toprule
Contrast & \multicolumn{1}{c}{Mean Difference Accuracy} & \multicolumn{1}{c}{95\% HDI}\\
\midrule
Object vs. Chance & -0.04 & [-0.13, 0.03]\\
Bullseye vs. Chance & -0.02 & [-0.12, 0.08]\\
ASL vs. Face & 0.13 & [0.04, 0.23]\\
Face vs. Object/Bullseye & 0.16 & [0.07, 0.24]\\
ASL vs. English & 0.23 & [0.17, 0.29]\\
\bottomrule
\end{tabular}
\end{threeparttable}
\end{center}
\end{table}

\begin{table}[h]
\begin{center}
\begin{threeparttable}
\caption{\label{tab:trio-rt-contrasts}Output of the model estimating differences in RT for specific contrasts of interest in Experiment 1.}
\begin{tabular}{lll}
\toprule
Contrast & \multicolumn{1}{c}{Mean Difference RT} & \multicolumn{1}{c}{95\% HDI}\\
\midrule
ASL vs. Face & 323.10 & [132.3, 522.6]\\
Face vs. Object/Bullseye & 408.20 & [286.6, 546.2]\\
ASL vs. English & 595.20 & [444.6, 760.8]\\
\bottomrule
\end{tabular}
\end{threeparttable}
\end{center}
\end{table}

\begin{table}[h]
\begin{center}
\begin{threeparttable}
\caption{\label{tab:trio-ewma-cuts}Output of the model estimating the point in the Reaction Time distribution when children's Exponentially Weighted Moving Average statistic crossed the pre-defined guessing threshold for the ASL and Face center stimulus types in Experiment 1.}
\begin{tabular}{lll}
\toprule
Center Stimulus Type & \multicolumn{1}{c}{Mean EWMA Cut Point} & \multicolumn{1}{c}{95\% HDI}\\
\midrule
ASL & 0.68 & [0.59, 0.78]\\
Face & 0.90 & [0.77, 1.03]\\
\bottomrule
\end{tabular}
\end{threeparttable}
\end{center}
\end{table}

\begin{table}[h]
\begin{center}
\begin{threeparttable}
\caption{\label{tab:trio-guess-cuts}Output of the model estimating the mean proportion of shifts categorized as language-driven by the Exponentially Weighted Moving Average model for the ASL and Face center stimulus types in Experiment 1.}
\begin{tabular}{lll}
\toprule
Center Stimulus Type & \multicolumn{1}{c}{Mean Language-driven} & \multicolumn{1}{c}{95\% HDI}\\
\midrule
Face & 0.34 & [0.23, 0.46]\\
Asl & 0.75 & [0.65, 0.84]\\
\bottomrule
\end{tabular}
\end{threeparttable}
\end{center}
\end{table}

\begin{table}[h]
\begin{center}
\begin{threeparttable}
\caption{\label{tab:trio-hddm}Summary of the Drift Diffusion Model output for the drift rate and boundary separation parameters for both all four center stimulus types in Experiment 1.}
\begin{tabular}{llll}
\toprule
Parameter & \multicolumn{1}{c}{Center Stim Type} & \multicolumn{1}{c}{Mean Param Estimate} & \multicolumn{1}{c}{95\% HDI}\\
\midrule
Boundary & Face & 1.34 & [1.21, 1.47]\\
Boundary & ASL & 1.76 & [1.65, 1.88]\\
Drift & Face & 0.55 & [0.3, 0.8]\\
Drift & ASL & 0.63 & [0.44, 0.82]\\
\bottomrule
\end{tabular}
\end{threeparttable}
\end{center}
\end{table}

\hypertarget{bayesian-data-analysis-models-experiment-2}{%
\section{Bayesian data analysis models: Experiment
2}\label{bayesian-data-analysis-models-experiment-2}}

In this appendix, we report the full output of the Bayesian linear mixed
effects models in Experiment 2.

\begin{table}[h]
\begin{center}
\begin{threeparttable}
\caption{\label{tab:noise-acc-model}Output of the logistic regression predicting accuracy as a function of noise condition and age group in Experiment 2.}
\begin{tabular}{llll}
\toprule
Noise Condition & \multicolumn{1}{c}{Age Group} & \multicolumn{1}{c}{Mean Accuracy} & \multicolumn{1}{c}{95\% HDI}\\
\midrule
Clear & children & 0.61 & [0.54, 0.68]\\
Noise & children & 0.67 & [0.6, 0.74]\\
Clear & adults & 0.90 & [0.87, 0.93]\\
Noise & adults & 0.92 & [0.89, 0.95]\\
\bottomrule
\end{tabular}
\end{threeparttable}
\end{center}
\end{table}

\begin{table}[h]
\begin{center}
\begin{threeparttable}
\caption{\label{tab:noise-rt-model}Output of the regression estimating reaction times (milliseconds) as a function of noise condition and age group in Experiment 2.}
\begin{tabular}{llll}
\toprule
Noise Condition & \multicolumn{1}{c}{Age Group} & \multicolumn{1}{c}{Mean RT} & \multicolumn{1}{c}{95\% HDI}\\
\midrule
Clear & children & 455.70 & [407, 503.6]\\
Noise & children & 500.20 & [446.6, 555.6]\\
Clear & adults & 542.40 & [486.2, 602.4]\\
Noise & adults & 595.20 & [532.4, 665.3]\\
\bottomrule
\end{tabular}
\end{threeparttable}
\end{center}
\end{table}

\begin{table}[h]
\begin{center}
\begin{threeparttable}
\caption{\label{tab:noise-guess-cuts}Output of the model estimating the mean proportion of shifts categorized as language-driven by the Exponentially Weighted Moving Average model for the each noise condition and age group in Experiment 2.}
\begin{tabular}{llll}
\toprule
Noise Condition & \multicolumn{1}{c}{Age Group} & \multicolumn{1}{c}{Mean Language-driven} & \multicolumn{1}{c}{95\% HDI}\\
\midrule
Clear & children & 0.36 & [0.32, 0.4]\\
Noise & children & 0.47 & [0.43, 0.51]\\
Clear & adults & 0.81 & [0.77, 0.86]\\
Noise & adults & 0.93 & [0.89, 0.97]\\
\bottomrule
\end{tabular}
\end{threeparttable}
\end{center}
\end{table}

\begin{table}[h]
\begin{center}
\begin{threeparttable}
\caption{\label{tab:noise-hddm}Summary of the Drift Diffusion Model output for the drift rate and boundary separation parameters for both processing contexts and age groups in Experiment 2.}
\begin{tabular}{llll}
\toprule
Parameter & \multicolumn{1}{c}{Age Group} & \multicolumn{1}{c}{Mean Parameter Estimate} & \multicolumn{1}{c}{95\% HDI}\\
\midrule
Drift & Children & 0.59 & [0.3, 0.89]\\
Drift & Adults & 1.86 & [1.48, 2.25]\\
Boundary & Children & 1.16 & [0.94, 1.39]\\
Boundary & Adults & 1.78 & [1.53, 2.05]\\
\bottomrule
\end{tabular}
\end{threeparttable}
\end{center}
\end{table}
\end{appendix}
